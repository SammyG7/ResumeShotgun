\documentclass{article}

\usepackage{booktabs}
\usepackage{tabularx}
\usepackage{hyperref}
\usepackage{nameref}
\usepackage{lipsum}

\title{SE 3XA3: Meeting Minutes\\The Resume Shotgun}

\author{Team 5, Proper Mars Tribe
		\\ Gavin Jameson, jamesong
		\\ Jeremy Langner, langnerj
		\\ Sam Gorman, gormans
}

\date{}

%\input{../Comments}

\begin{document}

\maketitle

\newpage

\tableofcontents

\newpage

This document outlines the group's meeting notes for every group meeting. 

\section{Considerations} \label{sec:considerations}
A formal "meeting" must be conducted over a call, meaning any messaging through text will not be included in this document. Meetings in which there were no major decisions will not be thoroughly documented. Cancelled meetings will still be noted in \nameref{sec:history}, but potential meetings that did not occur will not. Times will not be indicated and will be assumed to follow the \nameref{sec:plan} unless there are two meetings in a day or it is a meeting at a time not originally considered.

\section{Meeting Plan} \label{sec:plam}
Unless otherwise mentioned, our group will meet during the alotted lab sessions in person (Tuesday and Thursday, 9:30am-11:20am). If the group requires additional meetings, they will likely be conducted either Thursday from 5:00pm up until 10:00pm, or Friday from 4:30pm up until 11:00pm, as these are the only times outside of lab sessions all members should be consistently available. The designated scribe (alternates between members) will only need to actively take notes during major meetings.

\section{Meeting History} \label{sec:history}
%supersection

\subsection{Tuesday January 18, Teams}
Met group members. Started project selection.

\subsection{Thursday January 20, Teams}
Finalized project selection.

\subsection{Tuesday January 25, Teams}
Project was approved. Started problem statement and brainstormed ideas.

\subsection{Thursday January 27, Teams}
Completed problem statement.

\subsection{Tuesday February 1, In-Lab}
Completed lab 5-6 exercises.

\subsection{Thursday February 3, MSTeams}
Completed the development plan.

\subsection{Tuesday February 8, In-Lab}
Completed lab 7-8 exercises.

\subsection{Thursday February 10, Teams (9:30am)}
Started the SRS (Requirements Document).

\subsection{Thursday February 10, Teams (5:00pm)}
Completed the SRS (Requirements Document) revision 0.

\subsection{Tuesday February 15, In-Lab}
Completed lab 9-10 exercises.

\subsection{Thursday February 17, CANCELLED}
All members had completed all lab tasks for the week, and all members also had a midterm for another class later that day.

\subsection{Sunday February 27, Teams (6:00pm)}
Briefly discussed what changes from the large plan would be included for the demo deliverable. 

\subsection{Monday February 28, Teams (6:00pm)}
We decided to each work on a component to include in the demo. Gavin worked on a text UI, Jeremy worked on scraping for websites other than the original, and Sam worked on automatic logins and integration of the website scraping to UI.

\subsection{Tuesday March 1, In-Lab}
Work Period. 

\subsection{Thursday March 3, In-Lab}
Work Period. 

\subsection{Tuesday March 8, In-Lab}
Work Period. 

\subsection{Thursday March 10, In-Lab}
Work Period. 

\subsection{Tuesday March 15, In-Lab}
Work Period. 

\subsection{Thursday March 17, Teams}
Work Period. 

\subsection{Tuesday March 22, Thode}
Work period for demo.

\subsection{Thursday March 24, In-Lab}
Revision 0 demo.

\subsection{Tuesday March 29, In-Lab}
We finally planned on adding everything together (the menu was already added, but the PDF module and the link scraper was not yet). There were issues with combining the PDF parsing with the menu and saving to the profile, but after both interfaces were reviewed together it was not an issue. The link scraper was separate from other files and needed to be merged. 

The last main item that has not been implemented is the actual application process, which Jeremy took on, since he had worked on most of the other site interactions. Sam and Gavin will continue working on PDF and menu things, and can swap over to help when/if needed.

In part of the application process, as some jobs might need a cover letter, we planned out what would we done to manage cover letters:
\begin{itemize}
	\item Rename resume menu to document menu, and make original menu a submenu
	\item Add cover letter submenu, functionally identical to resume submenu
	\item File contains body of cover letter
	\item On job application, use provided body and job information (location, etc.) to create a pdf using latex
\end{itemize} 

Resolved (?) a lot of git issues for merging branche that had not been merged in awhile.

\newpage

\section{Revisions}

\begin{table}[hp]
\caption{Revision History} \label{tbl:rev}
\begin{tabularx}{\textwidth}{lX}
\toprule
\textbf{Date} & \textbf{Change}\\
\midrule
February 18& Created document, updated with past meetings (all minor) \\
February 28 & Added Tuesdays, added recent meetings \\
March 29 & Updated \\
\bottomrule
\end{tabularx}
\end{table}

\end{document}