\documentclass{article}

\usepackage{tabularx}
\usepackage{booktabs}

\title{SE 3XA3: Problem Statement\\The Resume Shotgun\\January 27, 2022}

\author{Team 5, Proper Mars Tribe
		\\ Gavin Jameson, jamesong
		\\ Jeremy Langner, langnerj
		\\ Sam Gorman, gormans
}

\date{}

%\input{../Comments}

\begin{document}

\maketitle

\newpage

\tableofcontents

\newpage

\section{Description}

\subsection{What Problem Are You Trying To Solve?}
Applying to jobs is a tedious and mundane task for most. Modern job search consists of searching nearly identical job posts posted by employers on a variety of job boards like Glassdoor, Indeed, LinkedIn, etc. These job postings describe the ideal and perfect candidate based on their qualifications, experience, skill, and characteristics; however, when applying, not all qualifications need to be fulfilled. \textit{The Resume Shotgun} will be an automated program that will remove the mundane and boring aspects of applying for jobs on these various job boards by searching, returning results and applying automatically. The program will require a resume, position title(s) and desired job boards to apply on. The program will then search for jobs based on the user's input and search for a sufficient match percentage between the job and user's qualifications. If the user achieves the threshold of matched qualifications the program will return a list of job positions (job title, company, link) for the user to read over and confirm their application. Once confirmed the program will apply to those jobs selected. Ultimately, The Resume Shotgun will use automation to efficiently find relevant job postings for the user and apply for them upon user confirmation.

\subsection{Why Is This An Important Problem?}
In our modern world, online job searching is becoming increasingly prevalent. 
Most students searching for Co-Op positions and adults entering the workforce have to work full time 
just to sift through hundreds of job postings searching for relevant applications. This work is tedious 
and unnecessary. The time that this process takes is time that many people simply don’t have.
The current tool partially alleviates this problem but still lacks a lot of functionality. 

Its scope is currently limited in which sites it functions on, and it doesn’t provide much 
customizability for the user. Allowing the user to add more meaningful filters to their search, 
along with bringing the tool to more relevant sites that the original developer missed will 
greatly increase the power of the program while keeping the core concept behind it the same.

\subsection{What Is The Context Of The Problem You Are Solving?}
The original project was created as part of a challenge called "5 Python Projects in 5 Days," where the solo developer was creating a new project each day. The original project was therefore created in one day, which follows that the code was limited to being a script to accomplish a single task due to time limitations. It was not originally meant to be particularly versatile or robust, nor as large as we hope to make it. The developer also said "I wanted to create a beginner-friendly collection of Python projects that are accessible, fun to learn, and easy to clone and expand" (in reference to the project we selected and the other four projects in the challenge). This helps us since there is already documentation on how to use the existing code, and it was made with the idea of future expansion.

The fact that it is a GitHub coding project and cited as being a "Software Engineering Job Application Bot" is what drew us to it, and the original script even has "Software Engineer" as a placeholder for the search term in the script. However, job searching is not a problem limited to Software Engineers, and there is no reason that this is essentially limited to being for them only. Although we may be Software Engineering Co-op students and it may benefit us to use, this is something that could help many students, or non-students, in similar job-searching positions.


\section{Revisions}

\begin{table}[hp]
\caption{Revision History} \label{TblRevisionHistory}
\begin{tabularx}{\textwidth}{llX}
\toprule
\textbf{Date} & \textbf{Developer(s)} & \textbf{Change}\\
\midrule
Jan 27 & Sam G & Section 1.2\\
Jan 27 & Gavin J & Formatting\\
Jan 27 & Gavin J & Section 1.3, Jeremy's 1.1\\
\bottomrule
\end{tabularx}
\end{table}



\iffalse
\wss{comment}
\ds{comment}
\mj{comment}
\cm{comment}
\mh{comment}
\fi

\end{document}
